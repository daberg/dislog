\documentclass[UTF8]{article}

\usepackage[left=1.5in, right=1.5in, top=1.2in, bottom=1.2in]{geometry}
\usepackage[nice]{nicefrac}
\usepackage{color}
\usepackage{graphicx}
\usepackage{listings}
\usepackage{mathtools}

\setlength\parindent{0pt}
\setlength{\parskip}{\baselineskip}

\definecolor{codeblue}{rgb}{0,0,0.5}
\definecolor{codegreen}{rgb}{0,0.5,0}
\definecolor{codeorange}{rgb}{0.9,0.6,0.1}
\definecolor{codered}{rgb}{0.6,0,0}

\lstset{
    language=Python,
    aboveskip=\baselineskip,
    belowskip=\baselineskip,
    basicstyle=\ttfamily,
    emphstyle=\bf\color{codered},
    keywordstyle=\color{codeblue},
    stringstyle=\color{codegreen},
    otherkeywords={True,False},
    emph={exhaustive, babygiant, pohlighellman, pollard}
}

\begin{document}

\title{
    Discrete logarithm algorithms \\
    \large Description and implementation
}
\author{Davide Bergamaschi}
\date{2019}
\maketitle

\begin{abstract}
\noindent In this document we present the cryptographic problem of discrete logarithm and a number of algorithms to solve it, along with our implementation of said algorithms in the Python programming language. In particular, the following algorithms will be examined: exhaustive search, Baby-Step Giant-Step, Pohlig-Hellman, Pollard's rho.
\end{abstract}

\section{The discrete logarithm problem}

Let $G$ be a cyclic group of order $n$, let $\alpha$ be a generator of G and let $\beta$ belong to G. The discrete logarithm of $\beta$ to the base $\alpha$ ($\log_{\alpha}\beta$) is the integer value $x$ such that $0 \leq x \leq n - 1$ and $\alpha^x=\beta$.

\section{Solving discrete logarithms}

Calculating discrete logarithms is in general a computationally hard problem. In the following section we present a selection algorithms that perform this task with different levels of sophistication.

Each algorithm is implemented by a Python function which takes $\alpha$, $\beta$ and $n$ as its first arguments, and returns either $\log_{\alpha}\beta$ or $None$ if the logarithm cannot be found due to input data not respecting preconditions (e.g. if $\alpha$ is not a generator).

\newpage

\subsection{Exhaustive search}

This naive method consists in computing $\alpha^0, \alpha^1, \ldots, \alpha^{n - 1}$ until $\beta$ is eventually obtained.

\begin{minipage}{\linewidth}
\lstinputlisting{lst1-exhaustive.py}
\end{minipage}

The time complexity is trivially given by $O(n)$ multiplications.

\subsection{Baby-Step Giant-Step}

This method relies on the fact that the logarithm $x$ can be written as $x = i m + j$, where $m$ can be conveniently chosen as $\lceil \sqrt{n} \rceil$. By doing so, one obtains $\beta = \alpha^{x} = \alpha^{i m + j} = \alpha^{i m} \alpha^{j}$, which is true if and only if $\beta (\alpha^{-m})^i = \alpha^{j}$.

The algorithm hence proceeds in the following way. For $0 \leq j < m$, entries $(a^j, j)$ are computed and stored in a hash table (hashed on the first component).

Then, for $0 \leq i < m$, $\beta (\alpha^{-m})^i$ is computed. At each iteration the algorithm checks if the obtained value is present in the hash table. Upon finding a match with a value $j$, the logarithm x is obtained as $x = i m + j$.

\begin{minipage}{\linewidth}
\lstinputlisting{lst2-babygiant.py}
\end{minipage}

In the first part of the implementation $m = \sqrt{n}$ is computed using a fixed precision arithmetics library to handle big numbers. In particular, the square root is calculated at least up to the first decimal place and then rounded up.

In each iteration of the last part, $\beta (\alpha^{-m})^i$ is computed by multiplying the $candidate$ variable, initially set to $\beta$, by the precomputed value $\alpha^{-m}$.

The time complexity is asymptotically dominated by the $O(\sqrt{n})$ initial multiplications, while the space complexity amounts to $\sqrt{n}$ group elements.

\subsection{Pohlig-Hellman algorithm}

This method leverages the prime factorization of the group order: $n = p_1^{e_1} \ldots p_r^{e_r}$.

For $1 \leq i \leq r$, the algorithm calculates $x_i = x \bmod{p_i^{e_i}}$, obtaining a set of congruences that can ultimately be solved with the Chinese Remainder Theorem to find $x$.

In particular, for each $p_i^{e_i}$, the algorithm calculates the $p_i$-ary representation of $x$:
$$x_i = x \bmod{p_i^{e_i}} = l_0 + l_1 p_i + l_2 p_i^2 + \ldots + l_{e_i - 1} p_i^{e_i - 1},$$
which allows for a further reduction of the magnitude of the calculations.

From the definition of modulus, there exists some integer $h$ for which $x=x_i+hp_i^{e_i}$. Since \mbox{$\beta=\alpha^x$}, one can derive:
\begin{align*}
    \beta^{n/p_i} & = (\alpha^x)^{n/p_i}                \\
                  & = (\alpha^{n/p_i})^x                \\
                  & = (\alpha^{n/p_i})^{x_i+hp_i^{e_i}} \\
                  & = (\alpha^{n/p_i})^{l_0 + l_1 p_i + l_2 p_i^2 + \ldots + l_{e_i - 1} p_i^{e_i - 1}+hp_i^{e_i}} \\
                  & = (\alpha^{n/p_i})^{l_0 + K p_i} \quad \text{\small{(for some integer $K$)}}  \\
                  & = (\alpha^{n/p_i})^{l_0} (\alpha^{n/p_i})^{K p_i} \\
                  & = (\alpha^{n/p_i})^{l_0} \quad \text{\small{(since $\alpha^{n/p_i}$ is of order $p_i$)}},
\end{align*}
from which $l_0$ can be found using any other discrete logarithm technique.

Then, starting from $\beta\alpha^{-l_0} = \alpha^{l_1 p_i + l_2 p_i^2 + \ldots + l_{e_i - 1} p_i^{e_i - 1}+hp_i^{e_i}}$ and by raising both sides to the power of $\nicefrac{n}{p_i^2}$, one can obtain $(\alpha^{n/p_i})^{l_1}=(\beta\alpha^{-l_0})^{n/p_i^2}$ and ultimately $l_1$ in a similar way.

The remaining digits can be computed analogously. The calculations performed for each $p_i^{e_i}$ in the first part of the algorithm are in conclusion the following:
\begin{align*}
    l_0         &              = \log_{\alpha^{n / p_i}}(\beta)^{n / p_i}                              \\
    l_1         &              = \log_{\alpha^{n / p_i}}(\beta\alpha^{-l_0})^{n / p_i^2}               \\
    l_2         &              = \log_{\alpha^{n / p_i}}(\beta\alpha^{-l_0 - l_1 \, p_i})^{n / p_i^3}  \\
                & \vdotswithin{=}                                                                      \\
    l_{e_i - 1} &              = \log_{\alpha^{n / p_i}}(\beta\alpha^{-l_0 - l_1 \, p_i - \, \ldots \, - l_{e_i - 2} \ p_i^{e_i - 2}})^{n / p_i^{e_i}}.
\end{align*}

Determining all the $x_i$ values yields a set of congruences of the following type:
\begin{align*}
    x & \equiv x_1 \bmod{p_1^{e_1}} \\
      & \vdotswithin{\equiv}        \\
    x & \equiv x_r \bmod{p_r^{e_r}},
\end{align*}
where the moduli are pairwise coprime by construction. The Chinese Remainder Theorem hence guarantees the existence of a solution $x$, which can be found with any suitable computational method (e.g. Gauss's algorithm).

\begin{minipage}{\linewidth}
\lstinputlisting{lst3-pohlighellman.py}
\end{minipage}

The prime factorization of $n$ is passed as fourth argument in the form of a Python dictionary, with keys being the prime factors and values are respectively their exponents.

Notice that during the computation of an $x_i$ value, at each iteration $j$ in the internal loop the implementation keeps track of the necessary powers of $p_i$ ($p_i^{j-1}$, $p_i^j$ and $p_i^{j+1}$), so that only one multiplication is enough to calculate them at each step.

The running time is given by $O(\sum_{i=1}^{r}e_i(\lg n + \sqrt{p_i}))$.

\end{document}
